\section{Declarative Programmable Storage}
\label{sec:prog-model}

Current ad-hoc approaches to programmable storage restrict use to developers
with distributed programming expertise, knowledge of the intricacies of the
undelrying storage system and its performance model, and primarily use
hard-coded imperative methods. This restricts the use of optimizations that
can be performed automatically or derived from static analysis.  Based on the
challenges we have demonstrated stemming from the dynamic nature and large
design space of programmable storage, we present an alternative, declarative
programming model that can reduce the learning curve for new users, and allow
existing developers to increase producitivty by writing less code that is more
portable.

The model we propose corresponds to a subset of the Bloom language which a
declarative language for expressing distributed programs as an unordered set
of rules~\cite{alvaro:cidr11}. These rules fully specify program semantics and
allow a programmer to ignore the details associated with how a program is
evaluated. This level of abstraction is attractive for building storage
interfaces whose portability and correctness is critical. We model the
storage system state uniformly as a collection of relations, with interfaces
being expressed as a collection of \emph{queries} over a request stream that
are filtered, transformed, and combined with system state. Next we present a
brief example of the CORFU shared-log interface expressed using this model.

\subsection{Example: The CORFU Log Interface}

The CORFU log protocol achieves high-performance in part by striping the log
across a large number of fast storage devices. Using our declarative language
we model the log as a single relation, hiding the implementation detail of log
partitioning. Lines XY in Listing~\ref{lst:state} show the declaration of
state for the CORFU interface consisting of two persistent collections: one
for the log data, and one for interface metadata. The mapping of collections
onto physical storage is abstracted at this level, permitting optimizations
discussed earlier to be applied transparently.  Note that due to space
limitations we only highlight salient features, and the reader may refer
to~\cite{watkins:ucsc-soe-16-12} for a full program listing.

Lines XY define operations as scratch collections that are not persistent;
they contain state only for the duration of a single time step. Operations may
be further sub-divided for specific cases, such as \emph{valid} or
\emph{invalid} depending on if they reflect up-to-date system state.
An implementation of this guard may select to cache the epoch value in
volatile storage because it infrequently changes.

\begin{lstlisting}[caption={State Declaration}, label=lst:state]
state do
  table :epoch, [:epoch]
  table :log, [:pos] => [:state, :data]

  scratch :write_op, op.schema
  scratch :trim_op,  op.schema

  # op did or did not pass the epoch guard
  scratch :valid_op,   op.schema
  scratch :invalid_op, op.schema

  # epoch guard
  invalid_op <= (op * epoch).pairs{|o,e|
    o.epoch <= e.epoch}
  valid_op <= op.notin(invalid_op)
  ret <= invalid_op{|o|
    [o.type, o.pos, o.epoch, 'stale']}
end
\end{lstlisting}

The write-once 64-bit address space exposed by CORFU storage devices depends
on a fast lookup within a potentially large and sparse index. The declarative
specification shown in Listing~\ref{lst:write} enables an implementation to
select an index and storage method independently.

\begin{lstlisting}[caption={Write}, label=lst:write]
bloom :write do
  temp :valid_write <= write_op.notin(found_op)
  log <+ valid_write{ |o| [o.pos, 'valid', o.data]}
  ret <= valid_write{ |o|
    [o.type, o.pos, o.epoch, 'ok'] }
  ret <= write_op.notin(valid_write) {|o|
    [o.type, o.pos, o.epoch, 'read-only'] }
end
\end{lstlisting}

The CORFU interface depends on applications to mark portions of the log as
unused in order to facilitate garbage collection.  In Listing~\ref{lst:trim}
entries are tracked as unused for reclamation, and implementations may take
advantage of specific optimizations provided by an index implementation or
hardware support found in modern non-volatile memories.

\begin{lstlisting}[caption={Trim}, label=lst:trim]
bloom :trim do
  log <+- trim_op{|o| [o.pos, 'trimmed']}
  ret <=  trim_op{|o|
    [o.type, o.pos, o.epoch, 'ok']}
end
\end{lstlisting}

Amazingly, only a few code snippets can express the semantics of the entire
storage device interface requirements in CORFU. For reference our prototype
implementation of CORFU in Ceph (called
ZLog\footnote{https://github.com/noahdesu/zlog}) is written in C++ and the
storage interface component comprises nearly 700 lines of code, and uses a
hard-coded indexing strategy that has been rewritten multiple times to explore
alternative optimization techniques. Beyond the convenience of writing less
code, it is far easier for the programmer writing an interface such as CORFU
to convenience herself of the correctness of the high-level details of the
implementation without being distracted by issues related to physical design
or the many other gotchas that one must deal with when writing low-level
systems software.
