\section{Programming Model}
\label{sec:prog-model}

It is important to not restrict use to developers seasoned in distributed
programming and those with knowledge of the intricacies of Ceph and its
performance model. Any hard-coded imperative method for build storage
interfaces today restrict the optimizations that can be performed
automatically or derived from static analysis.  To this end we present a
declarative programming model that can reduce the learning curve for new
users, and allow existing developers to increase producitivty by writing less
code that is more portable.

The model corresponds to a subset of the Bloom language which a
declarative language for expressing distributed programs as an unordered set
of rules~\cite{alvaro:cidr11}. These rules fully specify program semantics and allow a programmer
to ignore the details associated with how a program is evaluated. This level
of abstraction is attractive for building storage interfaces whose portability
and correctness is critical.

We model the storage system state uniformly as a collection of relations. The
composition of a collection of existing interfaces is then expressed as a
collection of high-level \emph{queries} that describe how a stream of requests
(API calls) are filtered, transformed and combined with existing state to
define streams of outputs (API call returns as well as updates to system
state).  Separating the relational semantics of such compositions from details
of their physical implementations introduces a number of degrees of freedom,
similar to the ``data independence'' offered by database query languages.  The
choice of access methods (for example, whether to use a bytestream interface
or a key/value store), storage device classes (e.g., whether to use HDDs or
SDDs), physical layout details (e.g. a 1:1 or N:1 mapping) and execution plans
(e.g. operator ordering) can be postponed to execution time.  The optimal
choices for these physical design details are likely to change much more often
than the logical specification, freeing the interface designer from the need
to rewrite systems and device and interface characteristics change.

\subsection{The CORFU Log Interface}

To demonstrate the approach of using a declarative language, we show how the
CORFU shared-log protocol can be expressed independent of the interfaces and
features of the underlying storage system. We have included the salient
components of the specification, but due to space constraints refer to the
reader to~\cite{tr} for a full program listing.

Listing~\ref{lst:state} shows the declaration of state for the CORFU
interface.  Lines 1---3 define the schema of the two persistent collections
that hold the current epoch value, and the log contents. These collections are
mapped onto storage but abstract away the low-level interface Lines 5---6
define named collections for each operation type. The scratch type indicates
that the data is not persistent, and only remains in the collection for a
single execution time step. The remaining scratch collections are defined to
further subdivide the operations based on different properties which we'll
describe next.

\begin{lstlisting}[caption={State Declaration}, label=lst:state]
state do
  table :epoch, [:epoch]
  table :log, [:pos] => [:state, :data]

  scratch :write_op, op.schema
  scratch :trim_op,  op.schema

  # op did or did not pass the epoch guard
  scratch :valid_op,   op.schema
  scratch :invalid_op, op.schema
end
\end{lstlisting}

Every operation handled by a storage device in CORFU is guarded by comparing
the epoch in which it was dispatched with the current configured epoch.
Listing~\ref{lst:guard} shows how an operation is compared against the current
state. An implementation of this guard may select to cache the epoch value in
volatile storage because it infrequently changes.

\begin{lstlisting}[caption={Epoch Guard}, label=lst:guard]
state do
  # epoch guard
  invalid_op <= (op * epoch).pairs{|o,e|
    o.epoch <= e.epoch}
  valid_op <= op.notin(invalid_op)
  ret <= invalid_op{|o|
    [o.type, o.pos, o.epoch, 'stale']}
end
\end{lstlisting}

The write-once 64-bit address space exposed by CORFU storage devices depends
on a fast lookup within a potentially large and sparse index. The declarative
specification shown in Listing~\ref{lst:write} enables an implementation to
select an index and storage method independently.

\begin{lstlisting}[caption={Write}, label=lst:write]
bloom :write do
  temp :valid_write <= write_op.notin(found_op)
  log <+ valid_write{ |o| [o.pos, 'valid', o.data]}
  ret <= valid_write{ |o|
    [o.type, o.pos, o.epoch, 'ok'] }
  ret <= write_op.notin(valid_write) {|o|
    [o.type, o.pos, o.epoch, 'read-only'] }
end
\end{lstlisting}

The CORFU interface depends on applications to mark portions of the log as
unused in order to facilitate garbage collection.  In Listing~\ref{lst:trim}
entries are tracked as unused for reclamation, and implementations may take
advantage of specific optimizations provided by an index implementation or
hardware support found in modern non-volatile memories.

\begin{lstlisting}[caption={Trim}, label=lst:trim]
bloom :trim do
  log <+- trim_op{|o| [o.pos, 'trimmed']}
  ret <=  trim_op{|o|
    [o.type, o.pos, o.epoch, 'ok']}
end
\end{lstlisting}

Amazingly, these few code snippets can express the semantics of the entire
storage device interface requirements in CORFU. For reference our prototype
implementation of CORFU in Ceph (called
ZLog\footnote{https://github.com/noahdesu/zlog}) is written in C++ and the
storage interface component comprises nearly 700 lines of code. This version
was designed to store log entries and metadata in the key-value interface,
thus requiring a lengthy rewrite to realize the performance advantage
available by using the bytestream interface.
Furthermore, the complexity introduced by using the
bytestream interface would grow the amount of code written in the optimized
version.

But beyond the convenience of writing less code, it is far easier for the
programmer writing an interface such as CORFU to convenience herself of the
correctness of the high-level details of the implementation without being
distracted by issues related to physical design or the many other gotchas that
one must deal with when writing low-level systems software.
