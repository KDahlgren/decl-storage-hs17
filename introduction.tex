\section{Introduction}
\label{sec:intro}

% what is the problem?
As large-scale unified storage systems evolve to meet the requirements of
next-generation hardware and an increasingly diverse set of applications,
\emph{de jure} approaches of the past---based on standardized interfaces---are
giving way domain-specific interfaces and optimizations. While promising,
current approaches to co-design are based on ad-hoc strategies that are
untenable.

% why is it interesting and important?

fixed storage apis force applications to use external data management, or
duplicate complex error prone processes when the system doesnt meet their
needs. fear of vendor lock-in is beginning to subside with open-source.  by
exposing internal services applications can compose existing services to
support application requirements.  also we may see entirely new sotrage system
based on apsecific cases, often based on consitency requirements.

data is the most critical component, but interfaces are just as critical
because they define access. how do we manage this trend?

% why is it hard? why do naive approaches fail?

The design space is large, and also quite different than traditional database
optimization techniques handle.

be very large, and it can be difficult to choose a design that future proofs an
implementation against hardware and software upgrades




its hard because the design space is quite large. afterall, how long have we
been dealing with blocks and files.
the reason we wanted to move away from standard storage apis in the first
place is because the storage system wants to evolve. so we cant advocate a
fixed api. domain specific knowledge of both applications and storage
systems is needed then to optimize any particular instance of co-design.

% why hasn't it been solved before? what is wrong with previous solutions? how
% does mine differ?
Malacology~\cite{sevilla:eurosys17} is a recently proposed \emph{programmable}
storage system that exposes common sub-systems found in distributed storage
systems for reuse by applications, avoiding duplication of complex error-prone
services.

Malacology demonstrates a set of prinicples for exposing services, and
demonstrates with real world examples. While the exact form of these services
is not well-defined, we argue in this paper that in order for storage systems
to continue to evolve, managing interface change must be a fundamental
component of programmable storage.

Declarative specification is the way to go here.

% what are the key components of my approach and results? also include anh
% specific limitations.
