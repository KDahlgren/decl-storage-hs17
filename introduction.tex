\section{Introduction}
\label{sec:intro}

Popular storage systems support a diversity of storage abstractions thereby
providing important disaggregation benefits: such \emph{unified} storage
systems support standard file, block, and object abstractions so that the same
hardware can be used for a larger range and more flexible mix of applications
than having a separate  system for each abstraction.  As these large-scale
unified storage systems evolve to meet the requirements of an increasingly
diverse set of applications and next-generation hardware, \emph{de jure}
approaches of the past---based on standardized interfaces---are giving way to
domain-specific interfaces and optimizations. While promising, current
approaches to co-design are based on ad-hoc strategies that are untenable.

The standardization of the POSIX I/O interface has been a major success,
allowing application developers to avoid vendor lock-in, and storage system
designers to innovate independently.
However, large-scale storage systems have been
dominated by proprietary offerings, preventing exploration of alternative
interfaces. Recently we have seen an increase
in the number of special-purpose storage systems, including high-performance
open-source storage systems that are modifiable, enabling system changes
without fear of vendor lock-in. Unfortunately, evolving storage system
interfaces is a challenging task that involves domain expertise and requires
programmers to forfeit the protection from change afforded by narrow
interfaces.

Malacology~\cite{sevilla:eurosys17} is a recently proposed storage system that
advocates for an approach to co-design called \emph{programmable storage}. The
approach is based on exposing low-level functionality as reusable building
blocks, allowing developers to custom-fit their applications to take advantage
of the code-hardened capabilities of the underlying system and avoid
duplication of complex and error-prone services. By recombining existing
services in the Ceph storage system, Malacology demonstrated how two
real-world services could be constructed. Unfortunately, implementing
applications on top of a system like Malacology can be an ad-hoc process
that is difficult to reason about and manage.

Despite the powerful approach advocated by Malacology, it requires programmers
to navigate a complex design space, simultaneously addressing often orthogonal
concerns including functional correctness, performance, and fault-tolerance.
Worse still, the domain expertise required to build a performant interface can
be quickly lost because interface composition is sensitive to changes in the
underlying environment such as hardware and software upgrades as well as
evolving workloads common place in unified storage systems.

To address these challenges, we advocate for the use of high-level declarative
languages (e.g. Datalog) as means by which new storage system interfaces are
programmed. By specifying the functional behavior of a storage interface once
in a relational (or algebraic language), an optimizer can use a cost model to
explore a space of functionally equivalent physical implementations. Much like
query planning and optimization in database systems, this approach will
separate the concerns of correctness and performance, protecting applications
against changes. But despite the parallels with database systems, the design
space is quite different.

In this paper we demonstrate the challenge of programmable storage by showing
the sensitivity of domain-specific interfaces to changes in the underlying
system. We then show that the relational model is able to capture the
functional behavior of a popular shared-log service, and finally we explore
additional optimizations that can be utilized to expand the space of
possible implementations.
