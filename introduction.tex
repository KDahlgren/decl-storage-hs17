\section{Introduction}
\label{sec:intro}

Popular storage systems support diverse storage abstractions by
providing important disaggregation benefits. Instead of maintaining
a separate system for each abstraction, \emph{unified} storage
systems, in particular, support standard file, block, and object abstractions so the same
hardware can be used for a wider range and a more flexible mix of applications. 
As these large-scale unified storage systems evolve to meet the requirements 
of an increasingly diverse set of applications and next-generation hardware, \emph{de jure}
approaches of the past---based on standardized interfaces---are giving way to
domain-specific interfaces and optimizations. While promising, the ad-hoc strategies characteristic of 
current approaches to co-design are untenable.

The standardization of the POSIX I/O interface has been a major success. General adoption
allows application developers to avoid vendor lock-in and encourages storage system
designers to innovate independently. However, large-scale storage systems are generally dominated 
by proprietary offerings, preventing exploration of alternative
interfaces. An increase in the number of special-purpose storage systems characterizes recent history
in the field, including the emergence of high-performance, and highly modifiable, open-source storage systems, 
which enable system changes without of vendor lock-in. Unfortunately, evolving storage system
interfaces is a challenging task requiring domain expertise and predicated on the willingness of
programmers to forfeit the protection from change afforded by narrow
interfaces.

Malacology~\cite{sevilla:eurosys17} is a recently proposed storage system that
advocates for an approach to co-design called \emph{programmable storage}. The
approach is based on exposing low-level functionality as reusable building
blocks, allowing developers to custom-fit their applications to take advantage
of the code-hardened capabilities of the underlying system and avoid
duplication of complex and error-prone services. By recombining existing
services in the Ceph storage system~\cite{weil:osdi2006-ceph}, Malacology demonstrates how two
real-world services, a distributed shared-log and a file system metadata load
balancer, could be constructed using a `dirty-slate` approach. Unfortunately, implementing
applications on top of a system like Malacology can be an ad-hoc process
that is difficult to reason about and manage.

Despite the powerful approach advocated by Malacology, the technique entails navigating 
a complex design space and simultaneously addressing often orthogonal
concerns (e.g. functional correctness, performance, and fault-tolerance).
Worse still, the availability of domain expertise required to build a performant interface is not a fixed or reliable resource. 
As a result, interface composition, as utilized in techniques such as the Malacology Approach, is sensitive to
the change, quickly evolving workloads, and software/hardware upgrades characteristic of unified storage systems.
This necessitates massive maintenance overheads.

To address these challenges, we advocate for the use of high-level declarative
languages (e.g. Datalog) as a means of programming new storage system
interfaces.  By specifying the functional behavior of a storage interface once
in a relational (or algebraic language), optimizers built around cost models
sensitive to storage and access tools and overheads can explore a space of
functionally equivalent physical implementations. Much like query planning and
optimization in database systems, this approach will logically differentiate
correctness and performance and protect higher-level services from lower-level
system changes. However, despite the parallels with database systems, this
paper demonstrates, and begins to address, fundamental differences in the
optimization design space.

In this paper we demonstrate the challenge of programmable storage by showing
the sensitivity of domain-specific interfaces to changes in the underlying
system. We then show that the relational model is able to capture the
functional behavior of a popular shared-log service, and finally we explore
additional optimizations capable of expanding the space of
possible implementations.
